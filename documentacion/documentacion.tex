% !TeX spellcheck = es_ES
\documentclass[a4paper, 12pt, spanish]{article}
\usepackage[spanish]{babel}
\selectlanguage{spanish}
\usepackage[utf8]{inputenc} %Se usa para que admita texto con tildes
\usepackage[T1]{fontenc}	%Se usa para que admita texto con tildes
\usepackage{amsmath}
\usepackage{listings}
\usepackage{color}	%Se usa para definir colores en el documento
\usepackage{ulem}	%Se usa para poner código en el documento.
\usepackage{bm}	%Se usa para hacer expresiones matemáticas en negrita
\usepackage[a4paper,vmargin=2cm,hmargin=1cm]{geometry}
\usepackage{diagbox}	%Se usa para hacer la linea diagonal dentro de la celda de una tabla

\usepackage{mathtools}	%se usa para hacer recuadros dentro de align con \Aboxed{}
\usepackage{chngcntr}	%se usa para resetear el contador de ecuaciones

\usepackage{empheq}		%se usa para hacer ecuaciones alineadas dentro de una caja.
\newcommand*\widefbox[1]{\fbox{\hspace{2em}#1\hspace{2em}}}

\usepackage{alltt}
\usepackage{pgfplots}
\usepackage{multicol}
\usepackage{caption}
\usepackage{subcaption}
\usepackage{float}

%configuracion de package {chngcntr}
\counterwithin*{equation}{section}
\counterwithin*{equation}{subsection}

%Definición de colores para el documento
\definecolor{dkgreen}{rgb}{0,0.6,0}
\definecolor{gray}{rgb}{0.5,0.5,0.5}
\definecolor{mauve}{rgb}{0.58,0,0.82}
%Definicion de ulem para el código
\lstset{
	language=PHP,
	aboveskip=3mm,
	belowskip=3mm,
	showstringspaces=false,
	columns=flexible,
	basicstyle={\ttfamily},
	numbers=left,
	numberstyle=\tiny\color{gray},
	keywordstyle=\color{blue},
	commentstyle=\color{dkgreen},
	stringstyle=\color{mauve},
	breaklines=true,
	breakatwhitespace=true,
	tabsize=3
}

\title{Secciones y Capítulos}
\author{Juan Aparicio}
\begin{document}
\begin{titlepage}
	\centering
	\includegraphics[width=0.15\textwidth]{imagenes/logo.png}\par\vspace{1cm}
	{\scshape\LARGE Taller de Informática\par}
	\vspace{1cm}
	{\scshape\Large Entrega\par}
	\vspace{1.5cm}
	{\huge\bfseries Documentación\par}
	\vspace{2cm}
	{\Large\itshape Juan Aparicio\par}
	\vfill
	supervisado por \par
	Carolina 		
	\vfill
	% Bottom of the page
	{\large 12, 2017\par}
\end{titlepage}

%Segunda página
\tableofcontents
\newpage

%Tercera página
%Acá van las secciones, etc

\section{Desgloce de Requerimientos}
\subsection{Registro usuario}
\begin{alltt}
SI ambas contraseñas son iguales
    SI la cédula no se encuentra ingresada
        inserto: nombre completo,
                 cédula,
                 fecha de nacimiento,
                 dirección,
                 teléfonos,
                 email,
                 sociedad medica,
                 emergencia móvil,
                 antecedentes de salud,
                 observaciones
    SINO
        ERROR: La cédula ya se encuentra ingresada en el sistema.
SINO
    ERROR: Las contraseñas no coinciden.
\end{alltt}

\subsection{Login de usuario}
\begin{alltt}
SI la cédula es numérica
    SI la cédula se encuentra ingresada
        SI hay un usuario con la cédula y la contraseña en la DB.
            Obtengo que tipo de usuario es de la DB.
            Redirecciono a la pagina apropiada (webmaster, socio, admin)
        SINO
            ERROR: contraseña incorrecta.
    SINO
    	ERROR: La cédula no se encuentra ingresada en el sistema.
SINO
	ERROR: La cédula tiene que tener un formato numérico.
\end{alltt}

\subsection{Ingreso de actividad}
\end{document}

